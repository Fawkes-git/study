%!TEX root = _thesis.tex
\thispagestyle{empty}
\begin{center}
	{\large 東京農工大学 工学府 情報工学専攻 2018年度 修士論文 要旨}\\
	\vspace{8mm}
	{\Large 題目 手指使用量の常時計測のためのウェアラブルデバイスの開発}\\
	\vspace{2mm}
	{\large Development of wearable device for continuous monitoring of finger usage}\\
	\vspace{8mm}
	{\large 学籍番号 17646137  氏名 松本 崇斗 (Takato MATSUMOTO)}\\
	\vspace{2mm}
	{\large 提出日 2019年1月31日}\\
	\vspace{4mm}
脳卒中麻痺リハビリテーションの目標は,食事,更衣,入浴などの日常生活動作ができるように患者の麻痺肢機能を改善することである.麻痺肢機能の改善を促進するリハビリテーションの介入方法を適切に評価するには,介入後の日常生活において,患者の麻痺肢使用量が増えたか否かといった,リハビリテーションの効果を定量的に測る手法が必要である.しかしながら,診療所や研究所で実施する検査では,質問紙やヒアリングによる調査が主体であり,日常生活における麻痺肢使用量を正確には評価することができない.例えば,片上肢麻痺患者の日常生活上での麻痺肢使用量を測る標準的手法としてMotor Activity Log(MAL)とAccelerometryがある.MALは,医師が患者に対し,麻痺肢使用の量と質について問う,質問形式の手法であり,測定結果が患者の主観的影響を受ける問題がある.そのため,客観的な測定手法が必要である.また,Accelerometryは,加速度計が埋め込まれた腕時計型の装置であり,上肢の使用量を測る手法であるが,加速度計に混入するノイズのため,指の使用量の測定には向かない.指の使用量を測定する手法として,Data GloveやMotion Capture Systemなどがあるが,これらの手法は,指の動きの阻害,コストが大きく空間的な制限がある,といった問題がある.さらに,磁力計と磁石の指輪を用いて指の使用量を測定するManumeterや,手の甲の皮膚の皺のパターンによって,指ジェスチャを識別するBehind The Palmといった手法が発表されているが,依然として指の使用量を測定する手法は確立していない.本研究では,日常生活下の上肢片麻痺患者の麻痺肢使用,特に指の使用量を測る手法を提案し,手指使用量の常時測定のためのウェアラブルデバイスの開発を目的とする.本研究の手指使用量の測定手法は,指の関節角度の変化が,指の使用を示すという仮定に基づく.関節角度の変化の推定には,ウェアラブルデバイスに搭載されたセンサを用いる.予備実験として健常者を対象に,手指を閉じた状態,開いた状態,人差し指と親指で輪を作った状態,計3つのジェスチャの識別が可能か調査した.5分割交差検証を行った結果,ジェスチャの平均正解率は98.9\%であった.結果から本手法により3つのジェスチャの識別が可能であることが示唆された.
\end{center}













