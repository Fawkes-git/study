%!TEX root = _thesis.tex
\chapter{結論と課題}

\section{結論}

リハビリテーション介入の効果を定量的に測定する手法として,手指使用量の常時計測のためのウェアラブルデバイスの開発を行った.本デバイスは赤外線距離センサを用いて指の第二関節角度を推定する指輪型のデバイスである.結果より,本手法により関節角度を推定でき,手指使用量を計測可能であることが示唆された.






指たくさんに付けれる
他の関節も測れる

\section{課題}
日常生活動作を一日中測定することは難しい.
・水に濡れる
・デバイスが有線であるため,邪魔になる可能性
エンジニアリングで解消できる.

Accelerometryでは,歩行時に腕を振る動作や,歩行により加速度計に加速度がかかり,ユーザがタスクを行なっていない時であっても,計測した加速度を腕の使用量と検知してしまう問題があった.
手指使用量を関節角度の総和と定義すると,何かに当たった時に動いた場合や,力を入れて使っているが,ほとんど関節角度の変化がない場合などを検出することができない.


再現性のテスト
SISやMALとの相関を見る