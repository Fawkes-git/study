\chapter{考察と課題}

\section{結論}
リハビリテーション介入の効果を定量的に測定する手法として,手指使用量の常時計測のためのウェアラブルデバイスの開発を行った.本デバイスは赤外線距離センサを用いて指の第二関節角度を推定する指輪型のデバイスである.結果より,本手法により関節角度を推定でき,手指使用量を計測可能であることが示唆された.

