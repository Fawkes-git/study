%!TEX root = _thesis.tex
\chapter{結論と課題}

\section{結論}
リハビリテーション介入の効果を定量的に測定する手法として,手指使用量の常時計測のためのウェアラブルデバイスの開発を行った.本デバイスは赤外線距離センサを用いて指の第二関節角度を推定する指輪型のデバイスである.結果より,本手法により関節角度を推定でき,手指使用量を計測可能であることが示唆された.



\section{課題}
本研究では手指使用量を関節角度変化の総和と定義している.
この定義の問題点は,指が何かに当たった時に動いた場合,指の使用を誤計測してしまうことである.
また,力を入れて指を使っているがほとんど関節角度の変化がない場合の
指の使用を測定することができない点である.

本デバイスは,有線であり,指輪部と腕輪部をワイヤで接続している.
また,赤外線距離センサを構成している,発光ダイオードとフォトトランジスタが大きい.
そのため,日常生活上でユーザが本デバイスを利用する際,ユーザビリティが高くない問題点がある.
さらに,日常生活では,水で手を洗うことが考えられ,本デバイスは防水性がないために,手を洗う際には取り外す必要があり,ユーザに負担をかける問題がある.
しかし,これらの問題点は,エンジニアリングで解消が可能である.
リング型デバイスのŌURA\cite{DeZambotti2017}は,センシングデータを無線でスマートフォンなどへ送信する機能や防水性を持っている.
また,赤外線距離センサを搭載しているが,その大きさは3mmほどであり,ユーザビリティを損なわない.
ŌURAの全体の大きさは,厚さ2.55mm,幅18mmで,重さは6gである.
ŌURAはユーザビリティが高く日常生活上で生体計測が可能である.
これらの理由から,エンジニアリングによって
ユーザビリティを損なわず本研究の手法を用い日常生活で指の使用量を計測するデバイスの開発は十分実現可能であると言える.
また,本デバイスはジェスチャ認識や関節角度の変化の認識が可能であるため,
指使用量の測定だけでなく,リモートコントローラとしての使用が期待できる.






