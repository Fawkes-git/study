%!TEX root = _thesis.tex
\chapter{序論}

\section{はじめに}
脳卒中麻痺リハビリテーションの目標は,食事,更衣,入浴などの日常生活動作ができるように患者の麻痺肢機能を改善することである.麻痺肢機能の改善を促進する介入方法の有効性を評価するためには,介入後の日常生活において,患者の麻痺肢使用量が実際に増えたか否かについて,リハビリテーションの効果を定量的に測る手法が必要である.しかしながら,病院やリハビリ施設で実施する検査では,質問紙やヒアリングによる調査が主体であり,日常生活下における麻痺肢使用量を正確に評価することができない\cite{Taub2006,Rand2009}.さらに,診療所や研究所で行われるテストでは,日常生活上での患者の麻痺肢使用量を計測することができない.現在,日常生活上の麻痺肢使用量を測定する手法として,Accelerometryが一般的である.この手法は加速度センサを用いるものであるが,手指の使用量を正確に計測,評価することはできない.そのため,本研究では日常生活において,麻痺患者の麻痺肢使用,特に,手指の使用量を計測,評価する手法を提案する.

\section{日常生活動作(ADL)}
日常生活動作(ADL)とは起居動作,移動,食事,更衣,排泄,入浴,整容など,日常生活を営む上で不可欠な基本動作のことである.
上肢麻痺患者の多くは,日常生活動作をするのに必要な上肢機能が健常者に比べて低いため,ADLが低くなる傾向がある\cite{Zeiler2017}.
ADLが低下すると患者の活動性が低下し,さらに身体機能の低下を招いてしまう\cite{Mlinac2016}.
リハビリテーションの目的は,ADLの低下を防ぎ,身体機能を改善することで,ADLが行えるようにすることである.
ADLの多くは上肢を使用するため,ADLの向上には,特に上肢機能の改善が不可欠である.
上肢を使用する動作の多くは,腕や手首,指といった様々な部位を使用するため,
これらの部位の総合的な回復が望まれる.

\section{上肢機能評価結果を用いた臨床判断}
臨床医やセラピストは,麻痺肢機能評価を元に患者の麻痺肢機能を治療,回復させる計画を立てる.
麻痺障害の段階や,麻痺肢機能によって患者に効果的なリハビリテーションの手法や時間が違う.
また,麻痺の回復は時間経過と関係があると示唆されており,
最も障害の回復が見込める期間は,障害が引き起こってから三カ月以内であると言われている.
さらに,麻痺の早期の障害の度合いは,最終的な運動障害に関係してくるため,
障害が生じてから,素早く麻痺障害の回復を促す必要がある\cite{Langhorne2011}.
これらの理由から,リハビリテーションの計画を立て,患者の麻痺障害を回復する上で麻痺肢機能評価は非常に重要である\cite{Lang2013}.

\section{診療所や研究所で行われる上肢機能の評価手法}
\subsection*{Action Research Arm Test}
Action Research Arm Test\cite{Hsieh2009,Lang2008,Lang2013,Lang2017,Nijland2010,VanDerPas2011,VanDerLee2004,Taub1998}は片側不全麻痺の患者の上肢機能評価手法である
.この手法は19項目を握る,把持,ピンチ,全身運動の四つのサブスケールに
分け評価する手法である.上肢機能は各項目ごと0から3の四段階で評価され,各項目の評価が3であり,合計評価が57である場合,健常者と同等の上肢機能であると評価される.
\begin{figure}[H]
\begin{center}
\begin{tabular}{cc}
\subfigure[ARAT Kit]{
\includegraphics[scale=0.3]{fig/ch1/ARAT_kit}
} &
\subfigure[ARAT Score Sheet]{
\includegraphics[scale=0.3]{fig/ch1/ARAT_SCORING}
} \\
\end{tabular}
\end{center}
   \caption{Action Research Arm Test\cite{Hsieh2009}}
\label{fig:ARAT}
\end{figure}


\subsection*{Box and Block Test}
Box and Block Test\cite{T.2005,Mathiowetz1985,Desrosiers1993,Lin2010}は実施が簡単かつ実施時間が短い上肢機能評価手法であり,ブロックの把持,移動,解放動作を評価する手法である.
被験者は,1分間で多くのブロックを箱から別の箱に移動させることを指示される.この手法では,上肢機能の能力を1分間に動かしたブロックの個数で評価する.
\begin{figure}[H]
  \centering
  \includegraphics[width=0.8\linewidth]{fig/ch1/babt}
  \caption{Block and Block Test\cite{T.2005}}
  \label{fig:babt}
\end{figure}

\subsection*{Chedoke Arm and Hand Activity Inventory}
Chedoke Arm and Hand Activity Inventory\cite{Barreca2004,Barreca2006,Barreca2005,Johnson2017}は腕,手の麻痺障害後の回復度合いを評価する指標である.
両手の使用を必要とするタスク13項目により,評価を行う.項目ごとに1$\sim$7の7段階で評価され,1は7の25\%以下の腕,手のパフォーマンスを意味する.
よって,高いスコアは,高い,腕や手のパフォーマンスを示す.タスクを行う時間が患者への負担となる場合,この手法を短時間で実施する手法として,Chedoke Arm and Hand Activity Inventory-9, Chedoke Arm and Hand Activity Inventory-8やChedoke Arm and Hand Activity Inventory-7といった手法がある.

\subsection*{Nine-Hole Peg Test}
Nine-Hole Peg Test\cite{Mathiowetz1985,Croarkin2004,Sunderland1989}
は手の器用さを測る簡単な手法である.被験者はペグをホールに刺し,刺した九つのペグを全て抜き取ることを指示される.テストの開始からペグを全て取り終えるまでの時間を計測し,その時間により,被験者の上肢機能を評価する.
\begin{figure}[H]
  \centering
  \includegraphics[width=0.8\linewidth]{fig/ch1/nine}
  \caption{Nine-Hole Peg Test\cite{Mathiowetz}}
  \label{fig:nine}
\end{figure}

\subsection*{Wolf Motor Function Test}
Wolf Motor Function Test\cite{Lin2010,Hsieh2009,Shima2009,Nijland2010}
は15個のタスクを含む上肢機能と患者の活動度合いを評価する手法である.1$\sim$6のタスクは関節の動きについて,7$\sim$15のタスクは総合的な上肢機能の評価項目である.患者は時間内にタスクをどれだけ遂行できたか(Functional Ability Score)で評価され,0はタスクを全く完了できないことを意味し,5はタスクを完全に完了できたことを意味する.時間での評価では,短い時間でタスクを完了できた場合,高い上肢機能であると評価する.

上記の診療所や研究所で行われる上肢機能の評価手法は,患者の麻痺肢機能を診療所や研究所でいずれかのテストを行うことにより評価しているため,
テストの実施により患者に負担がかかる問題や,日常生活上での患者の麻痺肢使用量を計測することができない問題がある.

\section{セルフレポート評価手法}
\subsection*{Motor Activity Log}
片上肢麻痺患者の日常生活上での麻痺肢使用量を測る標準的手法としてMotor Activity Log(MAL)\cite{Taub2006,Uswatte2005,Uswatte2000,Winstein2003,Michielsen2012}がある.MALは,医師が患者に対し,麻痺肢使用の量と質について直接問う,質問形式の手法である.MALの質問項目は,一般的な日常生活動作に関する質問である.日常生活動作とは,電気のスイッチをオンにする動作や引き出しを開ける動作といった,日常生活を営む上で最低限必要な動作である.MALの評価用紙をFig.\ref{fig:Motor Activity Log}に示す.MALの質問項目は30項目あり,日常生活動作時の麻痺肢使用の量と質について11段階の評価で患者が答える.研究所や診療所で行われるリハビリテーションテストは正確に患者の日常生活上での上肢機能を測定できない問題があるが,MALは上記の問題を解決できる上肢機能測定評価の標準的な手法の一つである.

\begin{figure}[H]
\begin{center}
\subfigure[Score sheet]{
\includegraphics[scale=0.3]{fig/ch1/mal3}
}
\begin{tabular}{cc}
\subfigure[Amout Scale]{
\includegraphics[scale=0.3]{fig/ch1/amount}
} &
\subfigure[How Well Scale]{
\includegraphics[scale=0.3]{fig/ch1/how}
} \\
\end{tabular}
\end{center}
   \caption{Motor Activity Log\cite{Taub2006}}
\label{fig:Motor Activity Log}
\end{figure}

MALは,質問形式といった測定結果が患者の認知レベルによる影響や質問者の主観的影響を受ける問題がある.例えばHow wellスケールの質問では,患者に対しあるタスクをどれほど良くできたかを問う.タスクを良くできたか悪くできたかは,患者の主観によって左右される.さらにAmountスケールの質問では,患者に対しあるタスクをどれほど多く,麻痺肢を使い行なったかを問う.この質問に答えるためにはタスク行なったかどうかを認知しておく必要がある.加えてMALは質問の回答に患者の記憶が依存するため,患者がタスクを行なったことを忘れてしまう問題がある他,患者に日常生活上でどんなタスクを行なったかを覚えておくよう指示しなければならず,患者に負担がかかる問題がある.これらの理由から,客観的かつ患者に負担がかかりにくい日常生活上での上肢機能測定手法が必要である.

\subsection*{Stroke Impact Scale}
Stroke Impact Scale(SIS)\cite{Lin2010b,Lin2010a,Lai2002,Duncan2001,Duncan2002,Duncan2003}は脳卒中のセルフレポート評価手法である.SISの評価項目は,強度,手の機能,ADL,IADL,モビリティ,コミュニケーション,感情,記憶と思考,
そして参加/役割機能の8つの分野に分かれている.患者は自身で5点のリッカート尺度を使用して各項目の能力を評価する.
手の機能とADL,IADLのサブスケールは,上肢機能の測定に最も関連性があるとされている.
各サブスケールのスコアは0から100の範囲であり,健常者のセルフレポートスコアは各サブスケールにおいて100で示される.
SISの実施は対面式のインタビュー形式,電話,または郵便により行われる.
郵便で行われた大規模なSISは被験者への負担が大きく,脳卒中の被験者の50\%が自分でアンケートを完成させることができない問題がある.

\section{定量的上肢使用量計測に関する研究}
\subsection*{Accelerometry}
Accelerometryは,加速度計が埋め込まれた腕時計型のウェアラブルデバイスで上肢の使用量を測る手法である.Accelerometry\cite{Chen2005,Hayward2016,Dwiputra2017,VanDerPas2011,VanDerLee2004,Thrane2011,Seitz2011}をFig.\ref{fig:Accelerometry}に示す.Accelerometryは麻痺肢の手首に装着することで,麻痺肢の使用量を測定する.Accelerometryは腕の使用量が多ければ,大きな加速度が記録され,腕の使用量が少なければ,記録される加速度は小さいといったことを
仮定したシンプルな腕の使用量の評価手法である.
日常生活動作とともに発生する,腕の加速度を記録する.データ記録装置とバッテリーが内蔵されているため,麻痺肢使用量の常時計測に向いている\cite{VanDerPas2011}.さらに,Accelerometryはリハビリ介入後の上肢使用量の増加,減少を示すことができることが示されている\cite{Uswatte2006}.
また,Accelerometryによる上肢機能の評価は,診療所や研究所で行われる上肢機能の評価と高い相関を持っている.Accelerometryでの評価と,診療所や研究所で行われる上肢機能評価との相関を以下のTable\ref{table:measure}に示す.

\begin{table}[H]
  \caption{Relationships Between Accelerometer Metrics and Measures}
  \label{table:measure}
  \centering
  \begin{tabular}{ll}
    \hline
    Measure & Correlation\\
    \hline \hline
    Action Research Arm Test &$r = 0.40-0.59$\cite{Lang2007,Rand2012}\\
    Wolf Motor Function Test &$r = 0.62$\cite{Lang2007}\\
    Box and Block Test &$r = 0.62$\cite{Rand2012}\\
    Motor Activity Log &$r = 0.52-0.91$\cite{Uswatte2000,Uswatte2005,Uswatte2006}\\
    Stroke Impact Scale &$r = 0.61$\cite{Rand2012}\\
    \hline
  \end{tabular}
\end{table}

Accelerometryで計測される加速度データはノイズを多く含み,信頼性の高いデータを得ることができない.加速度データに混入するノイズは,測定された加速度が所定の時間内に,閾値を超える場合にのみ,麻痺肢使用のスコアを増加させるといった手法の閾値フィルタを用いて低減することができる.このアプローチによって得られたスコアは日常生活において,Table.\ref{table:measure}に記すように,腕を動かした時間と高い相関を持つことが示されている.しかし,閾値フィルタを使用したノイズ低減を行った場合,加速度が閾値に達しない小さな手の動きが見落とされる可能性がある.また,加速度計は手首に装着されているため,手指の精密な動きを計測できない問題がある.これらの理由から,Accelerometryは指の使用量の測定には向かない\cite{Uswatte2000}.
また,Accelerometryはタスクを行なっているのか,歩行時に手を振っているのかということを見分けられない.歩行時に腕を振っていることを検知する手法もあるが,この手法では,足やつま先などに追加のAccelerometryを装着する必要がある\cite{Ullery2015}.
さらに,Accelerometryには,患者がどういった動作をしていたのかを見分けることが難しい問題がある\cite{Hayward2016}.
\begin{figure}[H]
  \centering
  \includegraphics[width=0.8\linewidth]{fig/ch1/acc}
  \caption{Accelerometry\cite{Chen2005}}
  \label{fig:Accelerometry}
\end{figure}



\subsection*{Data Glove}
Data glove\cite{Lin2018,Tarchanidis2003}は手袋型のセンシング機器である.
手袋の中に,曲げセンサまたは光ファイバーが内蔵されているタイプと慣性計測装置(IMUセンサ)が配置されているタイプが存在する.
手袋の指の部分に曲げセンサや光ファイバーが搭載されており,
曲げセンサが曲がったことによる抵抗値の変化や,光ファイバーが曲がったことによる光量の変化を計測し,指のトラッキングを行う.
IMUセンサが配置されているタイプは,IMUセンサで指の加速度や角加速度を計測し,指のトラッキングを行う.
Fig.\ref{fig:Data glove}はIMUセンサが手袋に配置されたタイプのData gloveである.
Data gloveは手全体を覆うセンサが存在しているため,センサによる指や手の動きの阻害や,
Data gloveの取り外しによる煩雑といった問題がある.
また,装着時には手を洗えないため,日常生活上での使用に向かない.

\begin{figure}[H]
  \centering
  \includegraphics[width=0.6\linewidth]{fig/ch1/data_glove}
  \caption{Data glove\cite{Lin2018}}
  \label{fig:Data glove}
\end{figure}

\subsection*{Motion Capture System}
Motion capture systemはカメラ画像や赤外線距離センサにより,人体の動きをデジタルデータとして取得するシステムである.手のモーショントラッキングに特化したモーションキャプチャシステムにLeap MotionやUbiHand,Digits\cite{Ahmad2006,Kim2012}がある.これらの手法は赤外線カメラによって,手の動きをトラッキングするシステムである.しかし,精度の問題から指の動きのような小さな動きを取得するのは難しい.また,モーションキャプチャシステムを構成する計測機器を常に持ち運んで運用することは難しい.そのため,持ち運びができ,精度よく指のトラッキングが行えるシステムが必要である.

\begin{figure}[H]
  \centering
  \includegraphics[width=0.6\linewidth]{fig/ch1/mcs}
  \caption{Motion Capture System(Leap Motion)}
  \label{fig:Motion Capture System}
\end{figure}

\subsection*{Manumeter}
Manumeter\cite{Friedman2014}は磁力計と磁石の指輪を用いて手首や指の使用量を測定する手法である.ManumeterをFig.\ref{fig:Manumeter}に示す.手首に取り付けてあるデバイスは磁力計,加速度計とデータ記録のためのストレージ,マイクロコンピュータで構成されている.指の動きを磁石と磁力計間の磁力変化により推定し,指の使用量を測定する.Manumeterは手首の橈屈,尺屈と掌屈,背屈,指の伸展と屈曲を識別する.Manumeterは,指の角度の推定値と正解の相関係数が$R^2=0.39-0.61$であり,指の使用量の測定精度が低いという問題点がある.また,磁石を用いた手法であるために,複数の指の角度計測を同時に行うことが難しい.
\begin{figure}[H]
  \centering
  \includegraphics[width=0.6\linewidth]{fig/ch1/manumeter}
  \caption{Manumeter\cite{Friedman2014}}
  \label{fig:Manumeter}
\end{figure}

\subsection*{Behind The Palm}
手の甲の皮膚の皺のパターンを認識することによって,指ジェスチャを識別するBehind The Palm\cite{Recognition2017}といった手法が発表されている.
99.5\%の確率で20個のジェスチャを認識することができると報告されているが,キャリブレーションによるユーザーへの負担が大きい.また,手の動作の認識はできないため,
日常生活上での指の使用量の計測には向かない.
\begin{figure}[H]
  \centering
  \includegraphics[width=0.6\linewidth]{fig/ch1/btp}
  \caption{Behind The Palm\cite{Recognition2017}}
  \label{fig:Behind The Palm}
\end{figure}

研究レベルではData gloveやGoniometer,Motion capture system\cite{Binh2014,Valtin2017,Chen2003,Ren2011}などが手首や手の使用量を測定するために使用される.しかしながら,これらの手法は指の動きの阻害,空間的な制限といった問題があるため,日常生活における長時間の常時計測には向いていない.

以上の評価手法の強みと日常生活上動作の計測における問題点をTable.\ref{table:strength}とTable.\ref{table:weakness}にまとめた.
\begin{table}[H]
  \caption{Strength of each measurement}
  \label{table:strength}
  \centering
  \begin{tabular}{ll}
    \hline
    上肢機能の評価手法 & 強み\\
    \hline \hline 
    診療所や研究所で行われるテスト  & 簡単に短い時間で評価可能  \\
    セルフレポート & 日常生活上での麻痺肢使用を測定できる \\
    Accelerometry  & 定量的に日常生活上での麻痺肢使用を測定可能,患者の動作を阻害しにくい \\
    Data Glove  & 定量的な手指の測定が可能  \\
    Motion Capture System  & 定量的,患者の動きを阻害しない  \\
    Manumeter  & 定量的な手指の測定が可能 \\
    Behind The Palm  & 多くのジェスチャの識別が可能 \\ 
    \hline
  \end{tabular}
\end{table}

\begin{table}[H]
  \caption{Problem of each measurement}
  \label{table:weakness}
  \centering
  \begin{tabular}{ll}
    \hline
    上肢機能の評価手法 & 日常生活上動作の計測における問題点 \\
    \hline \hline 
    診療所や研究所で行われるテスト   & 定量的でない,日常生活上の麻痺肢使用を測定できない \\
    セルフレポート & 患者の主観や記憶,認知レベルに評価が左右される,定量的でない\\
    Accelerometry   & ノイズのため,小さな動きや,手指の測定に向かない \\
    Data Glove  &  高価,手を手袋が覆うため,日常生活上の使用に向かない \\
    Motion Capture System   & 高価,空間的な制限がある,高い精度のトラッキングができない \\
    Manumeter  & 磁石を用いるため,複数の指の測定に向かない,精度が低い\\
    Behind The Palm   & キャリブレーションの患者への負担が大きい,ジェスチャのみの測定\\ 
    \hline
  \end{tabular}
\end{table}

\section{リング型デバイス}
本研究手法はリング型のデバイスによって,指の使用量を計測する手法であるが,
既存のリング型デバイスには,脈拍センサ,体温センサを用いた睡眠計測や,加速度センサを用いたジェスチャ認識を行うデバイスが存在する.
\subsection*{ŌURA}
ŌURA\cite{DeZambotti2017}はリング型の睡眠トラッカーである.ŌURAは脈拍をセンシングし,脈拍によって,睡眠の質を評価するデバイスである.
脈拍の計測には,光電脈拍法を利用している.光電脈拍法は550nm付近の赤外線を生体に向かって照射し,フォトトランジスタを用いて,生体内を反射した光を計測する手法である.ŌURAは,リングを装着した指に向かって光電脈拍法を適用し,脈拍の計測を行なっている.
ŌURAは96\%の精度で,ユーザの睡眠時間を脈拍から測定できる.
また,デバイスの大きさは,厚さ2.55mm,幅18mmで,重さは6gである.
フル充電にかかる時間は30-60分であり,3日間の連続使用が可能である.
日常生活上で生体トラッキングをするデバイスの中でもユーザビリティが高い.
\begin{figure}[H]
  \centering
  \includegraphics[width=0.4\linewidth]{fig/ch1/oura}
  \caption{ŌURA\cite{DeZambotti2017}}
  \label{fig:oura}
\end{figure}

\subsection*{Magic Ring}
Magic Ring\cite{Wang2013}は加速度センサが埋め込まれたリング型デバイスである.
加速度計測により,ユーザのジェスチャ認識行う手法である.この手法は
歩行,走行,食事,料理,歯磨き,洗顔,洋服を畳む,掃除,書き取りといったジェスチャをKNNモデルを使用することで,97.2\%の確率で識別が可能である.
\begin{figure}[H]
  \centering
  \includegraphics[width=0.6\linewidth]{fig/ch1/magicring}
  \caption{Magic Ring\cite{Wang2013}}
  \label{fig:magicring}
\end{figure}

\subsection*{iRing}
iRing\cite{Ogata2012}は,赤外線距離センサが埋め込まれたリング型デバイスである.
iRingはユーザのジェスチャ認識を目的とした手法である.
赤外線距離センサで,リングと皮膚の距離を計測し,リングが外部から押された時の力の大きさや
指を曲げた時の力の大きさを計測する.また,リングの回転角度を計測する手法である.
\begin{figure}[H]
  \centering
  \includegraphics[width=0.6\linewidth]{fig/ch1/iring}
  \caption{iRing\cite{Ogata2012}}
  \label{fig:iring}
\end{figure}

様々なリング型デバイスがある中で,指の使用量の測定精度が高いデバイスは未だに存在していない.
また,診療所や研究所で行われる上肢機能の評価手法では,上肢機能のパフォーマンス評価やセルフレポートが標準的であり,
定量的な指使用量の測定はなされていない.
日常生活上での腕の使用量を計測する最も一般的な手法であるAccelerometryは,加速度計を用いた腕の使用量の計測を主としており,
指の使用量は測定できない.これらの理由から,依然として日常生活下の指の使用量を常時計測する手法は確立していない.
本研究では,日常生活下の上肢片麻痺患者の麻痺肢使用,特に指の使用量を測る手法を提案し,手指使用量の常時測定のためのウェアラブルデバイスの開発を目的とする.


\section{本論文の構成}
本論文の構成を以下に記述する.第1章では本研究の背景と麻痺肢使用を評価する手法や既存の研究について紹介する.
第2章では,本研究で開発するデバイスの開発方法や,計測されたセンサデータの信号処理について記述する.
第3章では,実験方法や本デバイスの評価方法について記述する.
第4章では実験によって得られた結果と,結果に対する考察を記述する.
第5章では,本研究のまとめと課題を記述する.